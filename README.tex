\documentclass{article}

% Language setting
\usepackage[italian]{babel}

% Set page size and margins
\usepackage[a4paper,top=2cm,bottom=2cm,left=3cm,right=3cm,marginparwidth=1.75cm]{geometry}

% Useful packages
\usepackage{amsmath}
\usepackage{graphicx}
\usepackage[colorlinks=true, allcolors=blue]{hyperref}

\title{CryptoNews}
\author{Manuel Tanzi}

\begin{document}
\maketitle


\section{Introduzione}

CryptoNews è un sito per la visualizzazione di informazioni e statistiche inerenti al mondo delle cripto.
In particolare si possono accedere ai dati, con uno scarto di 15 minuti,resi pubblici dalle differenti piattaforme online come \href{https://it.finance.yahoo.com/}{yahoo finance}, raccolte in questo unico sito con una schematizzazione semplice ed intuitiva, alla quale non è richiesto nessun tipo di login.

E' possibile anche avere una visualizzazione più dettagliata della cryptovaluta in questione con tutti i dettagli di riferimento stampati a schermo tramite grafici e link target. 

\section{Come appare il sito}

\subsection{Barra laterale}
La barra laterale del sito presenta differnti pulsanti di navigazione come la \textbf{Homepage} data dal pulsante con il logo dell'università di parma.
Il pulsante \textbf{Dashboard} che porta l'utente in una tipica schermata a blocchi nella quale è possibile vedere le informazioni di maggior rilievo della giornata corrente.
\textbf{Cryptos} che mostrerà a schermo tutte le criptovalute con i loro valori attuali alla quale è poi possibile accederne ai dettagli con un semplice click.
Infine \textbf{News} che riportà alcuni titoli tra i siti più famosi di notiziari alla quale è possibile accerdervi ed espandere l notizia per saperne di più.

Un ulteriore bottone presente nella barra laterale è quello della \textbf{Dark mode} che eseguirà un cambio di colori della schermata.


Da come è possibile vedere nelle Figure \ref{fig:normale},  \ref{fig:notifiche}, \ref{fig:ridotta}, la barra può assumere differenti forme in base a come più si preferisce.
Inoltre dalla Figura \ref{fig:notifiche} si nota che può comparire un numero di notifica nella pagina di riferimento. Purtroppo questa configurazione è stata tolta per una limitazione del software delle API alla quale il sito fa riferimento.

\subsection{Schermata Pricipale}
Dalle immagini sottostanti \ref{fig:topCrypto}, \ref{fig:topNews} si nota come in questa schermata le informazioni sono riportate secondo una grafica molto semplice ed intuitiva e sono le informazioni messe in ordine per importanza o per influenza.
Si tendono, quindi a mostrare solamente le informazioni più rilevanti e per poi lasciare l'approfondimento di essi nelle loro pagine dedicate.


\begin{figure}[!htb]
\minipage{0.32\textwidth}
  \includegraphics[width=\linewidth]{1.png}
  \caption{Normale}\label{fig:normale}
\endminipage\hfill
\minipage{0.32\textwidth}
  \includegraphics[width=\linewidth]{2.png}
  \caption{Notifiche}\label{fig:notifiche}
\endminipage\hfill
\minipage{0.32\textwidth}%
  \includegraphics[width=\linewidth]{3.png}
  \caption{Ridotta}\label{fig:ridotta}
\endminipage
\end{figure}

\begin{figure}[!htb]
\minipage{\textwidth}
    \centering
  \includegraphics[width=\linewidth]{4.png}
  \caption{top 10 Crypto}\label{fig:topCrypto}
\endminipage\hfill

\minipage{\textwidth}
    \centering
  \includegraphics[width=\linewidth]{5.png}
  \caption{Latest News}\label{fig:topNews}
\endminipage\hfill
\end{figure}
\clearpage

\subsection{Dettagli e info}
Nella schermata dedicata alla visualizzazione dei dati e delle informazioni generali riguardanti una specifica criptovaluta scelta dall'utente è possibile vedere proiettata tramite grafico l'andamento nel tempo del prezzo della moneta e tramite un selettore è possibile sceglierne la qunatità di tempo desiderata come da Figura \ref{fig:graphic}. 
Scorrendo verso il basso si possono invece visualizzare altre informazioni utili  tra coi i link alle pagine di riferimento alla stessa moneta ed alla loro documentazione come da Figura \ref{fig:info} e Figura \ref{fig:links}.

\begin{figure}[!htb]
\minipage{0.4\textwidth}
    \centering
  \includegraphics[width=\linewidth]{10.png}
  \caption{Grafico}\label{fig:graphic}
\endminipage\hfill
\minipage{0.4\textwidth}
    \centering
  \includegraphics[width=\linewidth]{11.png}
  \caption{Statistiche}\label{fig:stats}
\endminipage\hfill
\minipage{0.4\textwidth}
    \centering
  \includegraphics[width=\linewidth]{12.png}
  \caption{Info}\label{fig:info}
\endminipage\hfill
\minipage{0.4\textwidth}
    \centering
  \includegraphics[width=\linewidth]{13.png}
  \caption{Links}\label{fig:links}
\endminipage\hfill
\end{figure}
\clearpage

\section{Tecnologie usate}

Il progetto viene realizzato utilizzando \href{https://it.reactjs.org/}{React} e
\href{https://nodejs.org/it/}{Node.js} con l'ausilio di visual studio code come supporto per programmare e caricare il tutto su git in maniera tale da riuscire a programmare su differenti computer mantendo sempre l'ultima versione realizzata.

\subsection{Librerie utilizzate}
Sono presenti differenti librerie grafiche come
\textbf{Bootstrap}, \textbf{Antd}, \textbf{ChartJs}.
Librerie di utilità come \textbf{Millify}, \textbf{Helmet},\textbf{Moment}.
Mentre invece per la parte dedicata alla gestione delle API utilizzo il toolkit di \textbf{ReduxJs}.
Per le varie icone invece vengono usate le \textbf{react-icon} e \textbf{ant-design/icon}.

\subsection{Siti web di appoggio}
Per procurare le informazioni faccio affidamento al sito \textbf{RapidAPI} che offre molti servizi gratuiti con una piccola limitazione sulle richieste giornaliere e mensili. 


Nello specifico utilizzo \url{https://rapidapi.com/Coinranking/api/coinranking1/}
per prendere le tutto ciò che mi serve per quanto riguarda le crypto valute.


\url{https://rapidapi.com/microsoft-azure-org-microsoft-cognitive-services/api/bing-news-search1/} viene utilizzato per procurarsi tutte quelle che sono le news riguardanti un generico argomento.


\end{document}