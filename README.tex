---
author:
- Manuel Tanzi
title: CryptoNews
---

# Introduzione

CryptoNews è un sito per la visualizzazione di informazioni e
statistiche inerenti al mondo delle cripto. In particolare si possono
accedere ai dati, con uno scarto di 15 minuti,resi pubblici dalle
differenti piattaforme online come [yahoo
finance](https://it.finance.yahoo.com/), raccolte in questo unico sito
con una schematizzazione semplice ed intuitiva, alla quale non è
richiesto nessun tipo di login.

E' possibile anche avere una visualizzazione più dettagliata della
cryptovaluta in questione con tutti i dettagli di riferimento stampati a
schermo tramite grafici e link target.

# Come appare il sito

## Barra laterale

La barra laterale del sito presenta differnti pulsanti di navigazione
come la **Homepage** data dal pulsante con il logo dell'università di
parma. Il pulsante **Dashboard** che porta l'utente in una tipica
schermata a blocchi nella quale è possibile vedere le informazioni di
maggior rilievo della giornata corrente. **Cryptos** che mostrerà a
schermo tutte le criptovalute con i loro valori attuali alla quale è poi
possibile accederne ai dettagli con un semplice click. Infine **News**
che riportà alcuni titoli tra i siti più famosi di notiziari alla quale
è possibile accerdervi ed espandere l notizia per saperne di più.

Un ulteriore bottone presente nella barra laterale è quello della **Dark
mode** che eseguirà un cambio di colori della schermata.

Da come è possibile vedere nelle Figure
[\[fig:normale\]](#fig:normale){reference-type="ref"
reference="fig:normale"},
[\[fig:notifiche\]](#fig:notifiche){reference-type="ref"
reference="fig:notifiche"}, [3](#fig:ridotta){reference-type="ref"
reference="fig:ridotta"}, la barra può assumere differenti forme in base
a come più si preferisce. Inoltre dalla Figura
[\[fig:notifiche\]](#fig:notifiche){reference-type="ref"
reference="fig:notifiche"} si nota che può comparire un numero di
notifica nella pagina di riferimento. Purtroppo questa configurazione è
stata tolta per una limitazione del software delle API alla quale il
sito fa riferimento.

## Schermata Pricipale

Dalle immagini sottostanti
[\[fig:topCrypto\]](#fig:topCrypto){reference-type="ref"
reference="fig:topCrypto"}, [5](#fig:topNews){reference-type="ref"
reference="fig:topNews"} si nota come in questa schermata le
informazioni sono riportate secondo una grafica molto semplice ed
intuitiva e sono le informazioni messe in ordine per importanza o per
influenza. Si tendono, quindi a mostrare solamente le informazioni più
rilevanti e per poi lasciare l'approfondimento di essi nelle loro pagine
dedicate.

[Ridotta]{#fig:ridotta .image .placeholder original-image-src="1.png"
original-image-title="fig:" width="\\linewidth"}

[Ridotta]{#fig:ridotta .image .placeholder original-image-src="2.png"
original-image-title="fig:" width="\\linewidth"}

[Ridotta]{#fig:ridotta .image .placeholder original-image-src="3.png"
original-image-title="fig:" width="\\linewidth"}

[Latest News]{#fig:topNews .image .placeholder
original-image-src="4.png" original-image-title="fig:"
width="\\linewidth"}

[Latest News]{#fig:topNews .image .placeholder
original-image-src="5.png" original-image-title="fig:"
width="\\linewidth"}

## Dettagli e info

Nella schermata dedicata alla visualizzazione dei dati e delle
informazioni generali riguardanti una specifica criptovaluta scelta
dall'utente è possibile vedere proiettata tramite grafico l'andamento
nel tempo del prezzo della moneta e tramite un selettore è possibile
sceglierne la qunatità di tempo desiderata come da Figura
[\[fig:graphic\]](#fig:graphic){reference-type="ref"
reference="fig:graphic"}. Scorrendo verso il basso si possono invece
visualizzare altre informazioni utili tra coi i link alle pagine di
riferimento alla stessa moneta ed alla loro documentazione come da
Figura [\[fig:info\]](#fig:info){reference-type="ref"
reference="fig:info"} e Figura [9](#fig:links){reference-type="ref"
reference="fig:links"}.

[Links]{#fig:links .image .placeholder original-image-src="10.png"
original-image-title="fig:" width="\\linewidth"}

[Links]{#fig:links .image .placeholder original-image-src="11.png"
original-image-title="fig:" width="\\linewidth"}

[Links]{#fig:links .image .placeholder original-image-src="12.png"
original-image-title="fig:" width="\\linewidth"}

[Links]{#fig:links .image .placeholder original-image-src="13.png"
original-image-title="fig:" width="\\linewidth"}

# Tecnologie usate

Il progetto viene realizzato utilizzando
[React](https://it.reactjs.org/) e [Node.js](https://nodejs.org/it/) con
l'ausilio di visual studio code come supporto per programmare e caricare
il tutto su git in maniera tale da riuscire a programmare su differenti
computer mantendo sempre l'ultima versione realizzata.

## Librerie utilizzate

Sono presenti differenti librerie grafiche come **Bootstrap**, **Antd**,
**ChartJs**. Librerie di utilità come **Millify**,
**Helmet**,**Moment**. Mentre invece per la parte dedicata alla gestione
delle API utilizzo il toolkit di **ReduxJs**. Per le varie icone invece
vengono usate le **react-icon** e **ant-design/icon**.

## Siti web di appoggio

Per procurare le informazioni faccio affidamento al sito **RapidAPI**
che offre molti servizi gratuiti con una piccola limitazione sulle
richieste giornaliere e mensili.

Nello specifico utilizzo
<https://rapidapi.com/Coinranking/api/coinranking1/> per prendere le
tutto ciò che mi serve per quanto riguarda le crypto valute.

<https://rapidapi.com/microsoft-azure-org-microsoft-cognitive-services/api/bing-news-search1/>
viene utilizzato per procurarsi tutte quelle che sono le news
riguardanti un generico argomento.
