\documentclass{article}

% Language setting
\usepackage[italian]{babel}

% Set page size and margins
\usepackage[a4paper,top=2cm,bottom=2cm,left=3cm,right=3cm,marginparwidth=1.75cm]{geometry}

% Useful packages
\usepackage{amsmath}
\usepackage{graphicx}
\usepackage[colorlinks=true, allcolors=blue]{hyperref}

\title{CryptoNews}
\author{Manuel Tanzi}

\begin{document}
\maketitle


\section{Introduzione}

CryptoNews è un sito per la visualizzazione di informazioni e statistiche inerenti al mondo delle cripto.
In particolare si possono accedere ai dati, con uno scarto di 15 minuti,resi pubblici dalle differenti piattaforme online come \href{https://it.finance.yahoo.com/}{yahoo finance}, raccolte in questo unico sito con una schematizzazione semplice ed intuitiva, alla quale non è richiesto nessun tipo di login.

E' possibile anche avere una visualizzazione più dettagliata della cryptovaluta in questione con tutti i dettagli di riferimento stampati a schermo tramite grafici e link target. 

\section{Come appare il sito}

\subsection{Barra laterale}
La barra laterale del sito presenta differnti pulsanti di navigazione come la \textbf{Homepage} data dal pulsante con il logo dell'università di parma.
Il pulsante \textbf{Dashboard} che porta l'utente in una tipica schermata a blocchi nella quale è possibile vedere le informazioni di maggior rilievo della giornata corrente.
\textbf{Cryptos} che mostrerà a schermo tutte le criptovalute con i loro valori attuali alla quale è poi possibile accederne ai dettagli con un semplice click.
Infine \textbf{News} che riportà alcuni titoli tra i siti più famosi di notiziari alla quale è possibile accerdervi ed espandere l notizia per saperne di più.

Un ulteriore bottone presente nella barra laterale è quello della \textbf{Dark mode} che eseguirà un cambio di colori della schermata.


Da come è possibile vedere nelle Figure \ref{fig:normale},  \ref{fig:notifiche}, \ref{fig:ridotta}, la barra può assumere differenti forme in base a come più si preferisce.
Inoltre dalla Figura \ref{fig:notifiche} si nota che può comparire un numero di notifica nella pagina di riferimento. Purtroppo questa configurazione è stata tolta per una limitazione del software delle API alla quale il sito fa riferimento.


\begin{figure}[!htb]
\minipage{0.32\textwidth}
  \includegraphics[width=\linewidth]{1.png}
  \caption{Normale}\label{fig:normale}
\endminipage\hfill
\minipage{0.32\textwidth}
  \includegraphics[width=\linewidth]{2.png}
  \caption{Notifiche}\label{fig:notifiche}
\endminipage\hfill
\minipage{0.32\textwidth}%
  \includegraphics[width=\linewidth]{3.png}
  \caption{Ridotta}\label{fig:ridotta}
\endminipage
\end{figure}

\begin{figure}[!htb]
\minipage{\textwidth}
    \centering
  \includegraphics[width=\linewidth]{4.png}
  \caption{top 10 Crypto}\label{fig:topCrypto}
\endminipage\hfill

\minipage{\textwidth}
    \centering
  \includegraphics[width=\linewidth]{5.png}
  \caption{top 10 Crypto}\label{fig:topCrypto}
\endminipage\hfill
\end{figure}



\subsection{Schermata Pricipale}



 \href{https://www.overleaf.com/learn}{help library} for more  \url{https://www.overleaf.com/contact}.

\bibliographystyle{alpha}
\bibliography{sample}

\end{document}